% Options for packages loaded elsewhere
\PassOptionsToPackage{unicode}{hyperref}
\PassOptionsToPackage{hyphens}{url}
%
\documentclass[
]{article}
\usepackage{amsmath,amssymb}
\usepackage{lmodern}
\usepackage{iftex}
\ifPDFTeX
  \usepackage[T1]{fontenc}
  \usepackage[utf8]{inputenc}
  \usepackage{textcomp} % provide euro and other symbols
\else % if luatex or xetex
  \usepackage{unicode-math}
  \defaultfontfeatures{Scale=MatchLowercase}
  \defaultfontfeatures[\rmfamily]{Ligatures=TeX,Scale=1}
\fi
% Use upquote if available, for straight quotes in verbatim environments
\IfFileExists{upquote.sty}{\usepackage{upquote}}{}
\IfFileExists{microtype.sty}{% use microtype if available
  \usepackage[]{microtype}
  \UseMicrotypeSet[protrusion]{basicmath} % disable protrusion for tt fonts
}{}
\makeatletter
\@ifundefined{KOMAClassName}{% if non-KOMA class
  \IfFileExists{parskip.sty}{%
    \usepackage{parskip}
  }{% else
    \setlength{\parindent}{0pt}
    \setlength{\parskip}{6pt plus 2pt minus 1pt}}
}{% if KOMA class
  \KOMAoptions{parskip=half}}
\makeatother
\usepackage{xcolor}
\usepackage[margin=1in]{geometry}
\usepackage{color}
\usepackage{fancyvrb}
\newcommand{\VerbBar}{|}
\newcommand{\VERB}{\Verb[commandchars=\\\{\}]}
\DefineVerbatimEnvironment{Highlighting}{Verbatim}{commandchars=\\\{\}}
% Add ',fontsize=\small' for more characters per line
\usepackage{framed}
\definecolor{shadecolor}{RGB}{248,248,248}
\newenvironment{Shaded}{\begin{snugshade}}{\end{snugshade}}
\newcommand{\AlertTok}[1]{\textcolor[rgb]{0.94,0.16,0.16}{#1}}
\newcommand{\AnnotationTok}[1]{\textcolor[rgb]{0.56,0.35,0.01}{\textbf{\textit{#1}}}}
\newcommand{\AttributeTok}[1]{\textcolor[rgb]{0.77,0.63,0.00}{#1}}
\newcommand{\BaseNTok}[1]{\textcolor[rgb]{0.00,0.00,0.81}{#1}}
\newcommand{\BuiltInTok}[1]{#1}
\newcommand{\CharTok}[1]{\textcolor[rgb]{0.31,0.60,0.02}{#1}}
\newcommand{\CommentTok}[1]{\textcolor[rgb]{0.56,0.35,0.01}{\textit{#1}}}
\newcommand{\CommentVarTok}[1]{\textcolor[rgb]{0.56,0.35,0.01}{\textbf{\textit{#1}}}}
\newcommand{\ConstantTok}[1]{\textcolor[rgb]{0.00,0.00,0.00}{#1}}
\newcommand{\ControlFlowTok}[1]{\textcolor[rgb]{0.13,0.29,0.53}{\textbf{#1}}}
\newcommand{\DataTypeTok}[1]{\textcolor[rgb]{0.13,0.29,0.53}{#1}}
\newcommand{\DecValTok}[1]{\textcolor[rgb]{0.00,0.00,0.81}{#1}}
\newcommand{\DocumentationTok}[1]{\textcolor[rgb]{0.56,0.35,0.01}{\textbf{\textit{#1}}}}
\newcommand{\ErrorTok}[1]{\textcolor[rgb]{0.64,0.00,0.00}{\textbf{#1}}}
\newcommand{\ExtensionTok}[1]{#1}
\newcommand{\FloatTok}[1]{\textcolor[rgb]{0.00,0.00,0.81}{#1}}
\newcommand{\FunctionTok}[1]{\textcolor[rgb]{0.00,0.00,0.00}{#1}}
\newcommand{\ImportTok}[1]{#1}
\newcommand{\InformationTok}[1]{\textcolor[rgb]{0.56,0.35,0.01}{\textbf{\textit{#1}}}}
\newcommand{\KeywordTok}[1]{\textcolor[rgb]{0.13,0.29,0.53}{\textbf{#1}}}
\newcommand{\NormalTok}[1]{#1}
\newcommand{\OperatorTok}[1]{\textcolor[rgb]{0.81,0.36,0.00}{\textbf{#1}}}
\newcommand{\OtherTok}[1]{\textcolor[rgb]{0.56,0.35,0.01}{#1}}
\newcommand{\PreprocessorTok}[1]{\textcolor[rgb]{0.56,0.35,0.01}{\textit{#1}}}
\newcommand{\RegionMarkerTok}[1]{#1}
\newcommand{\SpecialCharTok}[1]{\textcolor[rgb]{0.00,0.00,0.00}{#1}}
\newcommand{\SpecialStringTok}[1]{\textcolor[rgb]{0.31,0.60,0.02}{#1}}
\newcommand{\StringTok}[1]{\textcolor[rgb]{0.31,0.60,0.02}{#1}}
\newcommand{\VariableTok}[1]{\textcolor[rgb]{0.00,0.00,0.00}{#1}}
\newcommand{\VerbatimStringTok}[1]{\textcolor[rgb]{0.31,0.60,0.02}{#1}}
\newcommand{\WarningTok}[1]{\textcolor[rgb]{0.56,0.35,0.01}{\textbf{\textit{#1}}}}
\usepackage{graphicx}
\makeatletter
\def\maxwidth{\ifdim\Gin@nat@width>\linewidth\linewidth\else\Gin@nat@width\fi}
\def\maxheight{\ifdim\Gin@nat@height>\textheight\textheight\else\Gin@nat@height\fi}
\makeatother
% Scale images if necessary, so that they will not overflow the page
% margins by default, and it is still possible to overwrite the defaults
% using explicit options in \includegraphics[width, height, ...]{}
\setkeys{Gin}{width=\maxwidth,height=\maxheight,keepaspectratio}
% Set default figure placement to htbp
\makeatletter
\def\fps@figure{htbp}
\makeatother
\setlength{\emergencystretch}{3em} % prevent overfull lines
\providecommand{\tightlist}{%
  \setlength{\itemsep}{0pt}\setlength{\parskip}{0pt}}
\setcounter{secnumdepth}{-\maxdimen} % remove section numbering
\usepackage{booktabs}
\usepackage{longtable}
\usepackage{array}
\usepackage{multirow}
\usepackage{wrapfig}
\usepackage{float}
\usepackage{colortbl}
\usepackage{pdflscape}
\usepackage{tabu}
\usepackage{threeparttable}
\usepackage{threeparttablex}
\usepackage[normalem]{ulem}
\usepackage{makecell}
\usepackage{xcolor}
\ifLuaTeX
  \usepackage{selnolig}  % disable illegal ligatures
\fi
\IfFileExists{bookmark.sty}{\usepackage{bookmark}}{\usepackage{hyperref}}
\IfFileExists{xurl.sty}{\usepackage{xurl}}{} % add URL line breaks if available
\urlstyle{same} % disable monospaced font for URLs
\hypersetup{
  pdftitle={Project 1},
  pdfauthor={Name: Wesley Newcomb Partner:},
  hidelinks,
  pdfcreator={LaTeX via pandoc}}

\title{Project 1}
\author{Name: Wesley Newcomb Partner:}
\date{2023-03-26}

\begin{document}
\maketitle

\hypertarget{background}{%
\subsection{Background}\label{background}}

The World Health Organization has recently employed a new data science
initiative, \emph{CSIT-165}, that uses data science to characterize
pandemic diseases. \emph{CSIT-165} disseminates data driven analyses to
global decision makers.

\emph{CSIT-165} is a conglomerate comprised of two fabricated entities:
\emph{Global Health Union (GHU)} and \emph{Private Diagnostic
Laboratories (PDL)}. Your and your partner's role is to play a data
scientist from one of these two entities.

\hypertarget{data}{%
\subsection{Data}\label{data}}

\begin{quote}
\href{https://github.com/CSSEGISandData/COVID-19/tree/master/csse_covid_19_data/csse_covid_19_time_series}{2019
Novel Coronavirus COVID-19 (2019-nCoV) Data Repository by John Hopkins
CSSE} Data for 2019 Novel Coronavirus is operated by the John Hopkins
University Center for Systems Science and Engineering (JHU CSSE). Data
includes daily time series CSV summary tables, including confirmations,
recoveries, and deaths. Country/region are countries/regions hat conform
to World Health Organization (WHO). Lat and Long refer to coordinates
references for the user. Date fields are stored in MM/DD/YYYY format.
\end{quote}

For this project, we will use global data sets for COVID-19 associated
confirmations and deaths.

In order to download these files without cloning the whole repository,
enter one of two commands in your terminal depending on your machine:

Windows wget \url{https://address.to.data/goes/here.csv} Mac curl
\url{https://address.to.data/goes/here.csv} -O Instructions The World
Health Organization has recently employed a new data science initiative,
CSIT-165, that uses data science to characterize pandemic diseases.
CSIT-165 disseminates data driven analyses to global decision makers.

CSIT-165 is a conglomerate comprised of two fabricated entities: Global
Health Union (GHU) and Private Diagnostic Laboratories (PDL). Your and
your partner's role is to play a data scientist from one of these two
entities. Discuss with your partner to decide who will be part of GHU
and PDL.

Getting Started One project member per group must create a new
repository on GitHub. Initialize this repository with a readme.md file
that lists each member of the group. If your group decides to
collaborate using a centralized workflow (recommended), then the project
member that created the repository must declare their partners as
collaborators in GitHub. Each project member will clone this repository
onto their machine using RStudio. In RStudio, create a project from
version control with GitHub using the HTTP address of the repository
created by project member.

All project members must first contribute to analyses by uploading data
sets respective to the entity they belong to in the CSIT-165 initiative.
GHU project members provide time series data counting COVID-19 related
recoveries and and deaths. PDL project members provide time series data
counting COVID-19 related confirmations.

\hypertarget{project-objectives}{%
\subsection{Project Objectives}\label{project-objectives}}

\hypertarget{objectives}{%
\subsubsection{Objectives}\label{objectives}}

This project will encompass many of the lessons we have learned
throughout the course. RMarkdown files must be written such that each
time you render the document it will download the necessary data sets
for analysis. Please render the RMarkdown file the day it is due to
reflect the most recent data sets. With this added functionality, your
code must be able to analyze the datasets regardless of the date you
render your document.

\hypertarget{objective-1}{%
\subsubsection{Objective 1}\label{objective-1}}

CSIT-165's first objective is to determine where COVID-19 originated
from. Predict where the origin started based on the area with the
greatest number of confirmations and deaths on the first recorded day in
the data set. Show this is the origin using an if statement.

\begin{verbatim}
## Confirmed Dataset
\end{verbatim}

\begin{verbatim}
##   Province.State Country.Region X1.22.20
## 1          Hubei          China      444
## 2      Guangdong          China       26
## 3        Beijing          China       14
## 4       Zhejiang          China       10
## 5       Shanghai          China        9
## 6      Chongqing          China        6
\end{verbatim}

\begin{verbatim}
## Deaths Dataset
\end{verbatim}

\begin{verbatim}
##   Province.State Country.Region X1.22.20
## 1          Hubei          China       17
## 2                   Afghanistan        0
## 3                       Albania        0
## 4                       Algeria        0
## 5                       Andorra        0
## 6                        Angola        0
\end{verbatim}

\begin{verbatim}
## Recovered Dataset
\end{verbatim}

\begin{verbatim}
##   Province.State Country.Region X1.22.20
## 1          Hubei          China       28
## 2                      Thailand        2
## 3                   Afghanistan        0
## 4                       Albania        0
## 5                       Algeria        0
## 6                       Andorra        0
\end{verbatim}

\begin{verbatim}
## Hubei, China has the most confirmed cases on the first day. 
## Hubei, China has the most deaths from the virus on the first day. 
## Hubei, China has the most recovered cases from the virus on the first day.
\end{verbatim}

\begin{verbatim}
## Hubei, China is the most likely origin of the virus.
\end{verbatim}

\hypertarget{objective-2}{%
\subsubsection{Objective 2}\label{objective-2}}

Where is the most recent area to have a first confirmed case? To do
this, you will need to use a for loop, if statement, and subsets.

\begin{Shaded}
\begin{Highlighting}[]
\NormalTok{confirmed\_df}\OtherTok{\textless{}{-}}\FunctionTok{read.csv}\NormalTok{(}\StringTok{"data/time\_series\_covid19\_confirmed\_global.csv"}\NormalTok{, }
                       \AttributeTok{header=}\ConstantTok{TRUE}\NormalTok{, }\AttributeTok{stringsAsFactors=}\ConstantTok{FALSE}\NormalTok{)}
\NormalTok{recent\_df}\OtherTok{\textless{}{-}}\FunctionTok{arrange}\NormalTok{(confirmed\_df[confirmed\_df[,}\FunctionTok{ncol}\NormalTok{(confirmed\_df)}\SpecialCharTok{{-}}\DecValTok{1}\NormalTok{] }\SpecialCharTok{==} \DecValTok{0} 
                                \SpecialCharTok{\&}\NormalTok{ confirmed\_df[,}\FunctionTok{ncol}\NormalTok{(confirmed\_df)] }\SpecialCharTok{\textgreater{}} \DecValTok{0}\NormalTok{,])}
\NormalTok{i}\OtherTok{\textless{}{-}}\DecValTok{0}
\CommentTok{\# If there are no new cases today loop back to find most recent region}
\CommentTok{\# to have new cases}
\ControlFlowTok{if}\NormalTok{ (}\FunctionTok{nrow}\NormalTok{(recent\_df) }\SpecialCharTok{==} \DecValTok{0}\NormalTok{) \{}
  \ControlFlowTok{while}\NormalTok{ (}\FunctionTok{nrow}\NormalTok{(recent\_df) }\SpecialCharTok{==} \DecValTok{0}\NormalTok{) \{}
\NormalTok{    i}\OtherTok{\textless{}{-}}\NormalTok{i}\SpecialCharTok{+}\DecValTok{1}
\NormalTok{    recent\_df}\OtherTok{\textless{}{-}}\FunctionTok{arrange}\NormalTok{(confirmed\_df[confirmed\_df[,}\FunctionTok{ncol}\NormalTok{(confirmed\_df)}\SpecialCharTok{{-}}\DecValTok{1}\SpecialCharTok{{-}}\NormalTok{i] }\SpecialCharTok{==} \DecValTok{0} 
                                \SpecialCharTok{\&}\NormalTok{ confirmed\_df[,}\FunctionTok{ncol}\NormalTok{(confirmed\_df)}\SpecialCharTok{{-}}\NormalTok{i] }\SpecialCharTok{\textgreater{}} \DecValTok{0}\NormalTok{,])}
\NormalTok{  \}}
\NormalTok{\}}
\FunctionTok{head}\NormalTok{(}\FunctionTok{select}\NormalTok{(recent\_df, Province.State, Country.Region, }\FunctionTok{ncol}\NormalTok{(confirmed\_df)}\SpecialCharTok{{-}}\DecValTok{1}\SpecialCharTok{{-}}\NormalTok{i, }
            \FunctionTok{ncol}\NormalTok{(confirmed\_df)}\SpecialCharTok{{-}}\NormalTok{i))}
\end{Highlighting}
\end{Shaded}

\begin{verbatim}
##     Province.State Country.Region X7.19.22 X7.20.22
## 1 Pitcairn Islands United Kingdom        0        4
\end{verbatim}

\begin{Shaded}
\begin{Highlighting}[]
\CommentTok{\# Vector is small enough that loop is reasonable}
\ControlFlowTok{for}\NormalTok{(i }\ControlFlowTok{in} \DecValTok{1}\SpecialCharTok{:}\FunctionTok{nrow}\NormalTok{(recent\_df))}
\NormalTok{\{}
  \ControlFlowTok{if}\NormalTok{ (recent\_df[i,}\DecValTok{1}\NormalTok{] }\SpecialCharTok{==} \StringTok{""}\NormalTok{) \{}
    \FunctionTok{cat}\NormalTok{(recent\_df[i,}\DecValTok{2}\NormalTok{], }\StringTok{"has recently had their first confirmed case. }\SpecialCharTok{\textbackslash{}n}\StringTok{"}\NormalTok{)}
\NormalTok{  \} }\ControlFlowTok{else}\NormalTok{ \{}
    \ControlFlowTok{if}\NormalTok{ (recent\_df[i,}\DecValTok{2}\NormalTok{] }\SpecialCharTok{==} \StringTok{""}\NormalTok{) \{}
      \FunctionTok{cat}\NormalTok{(recent\_df[i,}\DecValTok{1}\NormalTok{], }\StringTok{"has recently had their first confirmed case. }\SpecialCharTok{\textbackslash{}n}\StringTok{"}\NormalTok{)}
\NormalTok{    \} }\ControlFlowTok{else}\NormalTok{ \{}
      \FunctionTok{cat}\NormalTok{(recent\_df[i,}\DecValTok{1}\NormalTok{], }\StringTok{", "}\NormalTok{, recent\_df[i,}\DecValTok{2}\NormalTok{], }
          \StringTok{" has recently had their first confirmed case. }\SpecialCharTok{\textbackslash{}n}\StringTok{"}\NormalTok{, }\AttributeTok{sep=}\StringTok{""}\NormalTok{)}
\NormalTok{    \}}
\NormalTok{  \}}
\NormalTok{\}}
\end{Highlighting}
\end{Shaded}

\begin{verbatim}
## Pitcairn Islands, United Kingdom has recently had their first confirmed case.
\end{verbatim}

\hypertarget{objective-3}{%
\subsubsection{Objective 3}\label{objective-3}}

How far away are the areas from objective 2 from where the first
confirmed case(s) occurred? Please provide answer(s) in terms of miles.
Use the function distm from the R package geosphere to calculate the
distance between two coordinates in meters (geosphere::distm). You will
need to convert the value returned by distm from meters to miles (this
conversion is simple and can be found online). Please use a table or
printed statement to describe what Province/State and Country/Region
first confirmed cases occurred as well as the distance (in miles) away
from the origin. Please print the following: \{recent region\} is
\{distance in miles\} away from \{origin city, origin country\}.

\begin{Shaded}
\begin{Highlighting}[]
\NormalTok{origin\_df}\OtherTok{\textless{}{-}}\FunctionTok{arrange}\NormalTok{(confirmed\_df, }\SpecialCharTok{{-}}\NormalTok{X1.}\FloatTok{22.20}\NormalTok{)[}\DecValTok{1}\NormalTok{,]}
\NormalTok{confirmed\_df}\OtherTok{\textless{}{-}}\FunctionTok{read.csv}\NormalTok{(}\StringTok{"data/time\_series\_covid19\_confirmed\_global.csv"}\NormalTok{, }
                       \AttributeTok{header=}\ConstantTok{TRUE}\NormalTok{, }\AttributeTok{stringsAsFactors=}\ConstantTok{FALSE}\NormalTok{)}
\NormalTok{recent\_df}\OtherTok{\textless{}{-}}\FunctionTok{arrange}\NormalTok{(confirmed\_df[confirmed\_df[,}\FunctionTok{ncol}\NormalTok{(confirmed\_df)}\SpecialCharTok{{-}}\DecValTok{1}\NormalTok{] }\SpecialCharTok{==} \DecValTok{0} 
                                \SpecialCharTok{\&}\NormalTok{ confirmed\_df[,}\FunctionTok{ncol}\NormalTok{(confirmed\_df)] }\SpecialCharTok{\textgreater{}} \DecValTok{0}\NormalTok{,])}
\NormalTok{i}\OtherTok{\textless{}{-}}\DecValTok{0}
\CommentTok{\# If there are no new cases today loop back to find most recent region}
\CommentTok{\# to have new cases}
\ControlFlowTok{if}\NormalTok{ (}\FunctionTok{nrow}\NormalTok{(recent\_df) }\SpecialCharTok{==} \DecValTok{0}\NormalTok{) \{}
  \ControlFlowTok{while}\NormalTok{ (}\FunctionTok{nrow}\NormalTok{(recent\_df) }\SpecialCharTok{==} \DecValTok{0}\NormalTok{) \{}
\NormalTok{    i}\OtherTok{\textless{}{-}}\NormalTok{i}\SpecialCharTok{+}\DecValTok{1}
\NormalTok{    recent\_df}\OtherTok{\textless{}{-}}\FunctionTok{arrange}\NormalTok{(confirmed\_df[confirmed\_df[,}\FunctionTok{ncol}\NormalTok{(confirmed\_df)}\SpecialCharTok{{-}}\DecValTok{1}\SpecialCharTok{{-}}\NormalTok{i] }\SpecialCharTok{==} \DecValTok{0} 
                                \SpecialCharTok{\&}\NormalTok{ confirmed\_df[,}\FunctionTok{ncol}\NormalTok{(confirmed\_df)}\SpecialCharTok{{-}}\NormalTok{i] }\SpecialCharTok{\textgreater{}} \DecValTok{0}\NormalTok{,])}
\NormalTok{  \}}
\NormalTok{\}}
\CommentTok{\# Compute distances from origin}
\NormalTok{distances}\OtherTok{\textless{}{-}}\FunctionTok{distm}\NormalTok{(}\FunctionTok{select}\NormalTok{(recent\_df, Long, Lat), }\FunctionTok{select}\NormalTok{(origin\_df, Long, Lat))}
\CommentTok{\# Convert from m to miles}
\NormalTok{distances}\OtherTok{\textless{}{-}}\NormalTok{distances }\SpecialCharTok{*} \FloatTok{0.00062137}
\CommentTok{\# Add distance from origin to dataframe and sort by distance}
\NormalTok{recent\_df}\SpecialCharTok{$}\NormalTok{distance}\OtherTok{\textless{}{-}}\NormalTok{distances[,}\DecValTok{1}\NormalTok{]}
\NormalTok{recent\_df}\OtherTok{\textless{}{-}}\FunctionTok{arrange}\NormalTok{(recent\_df, distance)}
\FunctionTok{head}\NormalTok{(}\FunctionTok{select}\NormalTok{(recent\_df, Province.State, Country.Region, Lat, Long, distance))}
\end{Highlighting}
\end{Shaded}

\begin{verbatim}
##     Province.State Country.Region      Lat      Long distance
## 1 Pitcairn Islands United Kingdom -24.3768 -128.3242  8746.96
\end{verbatim}

\begin{Shaded}
\begin{Highlighting}[]
\CommentTok{\# Vector is small enough that loop is reasonable}
\ControlFlowTok{for}\NormalTok{ (i }\ControlFlowTok{in} \DecValTok{1}\SpecialCharTok{:}\FunctionTok{nrow}\NormalTok{(recent\_df)) \{}
\NormalTok{  city}\OtherTok{\textless{}{-}}\NormalTok{recent\_df[i, }\StringTok{"Province.State"}\NormalTok{]}
  
  \CommentTok{\# If there is no city use country}
  \ControlFlowTok{if}\NormalTok{ (city }\SpecialCharTok{==} \StringTok{""}\NormalTok{) \{}
\NormalTok{    city}\OtherTok{\textless{}{-}}\NormalTok{recent\_df[i, }\StringTok{"Country.Region"}\NormalTok{]}
\NormalTok{  \}}
  
  \FunctionTok{cat}\NormalTok{(city, }\StringTok{"is"}\NormalTok{, recent\_df[i, }\StringTok{"distance"}\NormalTok{], }
      \StringTok{"miles away from the virus origin in"}\NormalTok{, }
      \FunctionTok{paste0}\NormalTok{(origin\_df[}\DecValTok{1}\NormalTok{, }\StringTok{"Province.State"}\NormalTok{], }\StringTok{","}\NormalTok{), }
      \FunctionTok{paste0}\NormalTok{(origin\_df[}\DecValTok{1}\NormalTok{, }\StringTok{"Country.Region"}\NormalTok{], }\StringTok{"."}\NormalTok{), }\StringTok{"}\SpecialCharTok{\textbackslash{}n}\StringTok{"}\NormalTok{)}
\NormalTok{\}}
\end{Highlighting}
\end{Shaded}

\begin{verbatim}
## Pitcairn Islands is 8746.96 miles away from the virus origin in Hubei, China.
\end{verbatim}

\hypertarget{objective-4}{%
\subsubsection{Objective 4}\label{objective-4}}

CSIT-165 characterizes diseases using risk scores. Risk scores are
calculated as the ratio of deaths to confirmations, that is
Riskscore=100×deathsconfirmations . Risk scores equal to 100 indicate
the highest risk while risk scores equal to 0 indicate the lowest risk.
Areas are characterized as being especially vulnerable to loss if they
have higher risk scores. For this assignment, exclude cruise ships
(hint: they have lat and long coordinates of 0 or NA in this data set,
filter this out before calculating risk scores). Which area of the world
currently has the lowest risk score (if more than one, display the one
with the most confirmations)? Which area of the world currently has the
highest risk score (if more than one, display the one with the most
confirmations)? How do risk scores in these areas compare to global risk
score? Why might it be helpful to calculate metrics like risk scores for
different areas of the world and what would their limitations be (what
assumptions does risk score make and what important variables might be
left out)?

\begin{Shaded}
\begin{Highlighting}[]
\CommentTok{\# Datasets respresent a cumlative sum by date, so last column represents }
\CommentTok{\# sumation for region}
\NormalTok{confirmed\_df}\OtherTok{\textless{}{-}}\FunctionTok{read.csv}\NormalTok{(}\StringTok{"data/time\_series\_covid19\_confirmed\_global.csv"}\NormalTok{, }
                       \AttributeTok{header=}\ConstantTok{TRUE}\NormalTok{, }\AttributeTok{stringsAsFactors=}\ConstantTok{FALSE}\NormalTok{)}
\NormalTok{confirmed\_df}\OtherTok{\textless{}{-}}\FunctionTok{select}\NormalTok{(confirmed\_df, Province.State, }
\NormalTok{                     Country.Region, }\FunctionTok{ncol}\NormalTok{(confirmed\_df))}
\FunctionTok{names}\NormalTok{(confirmed\_df)[}\DecValTok{3}\NormalTok{] }\OtherTok{\textless{}{-}} \StringTok{"confirmed"}
\NormalTok{deaths\_df}\OtherTok{\textless{}{-}}\FunctionTok{read.csv}\NormalTok{(}\StringTok{"data/time\_series\_covid19\_deaths\_global.csv"}\NormalTok{, }
                    \AttributeTok{header=}\ConstantTok{TRUE}\NormalTok{, }\AttributeTok{stringsAsFactors=}\ConstantTok{FALSE}\NormalTok{)}
\NormalTok{deaths\_df}\OtherTok{\textless{}{-}}\FunctionTok{select}\NormalTok{(deaths\_df, Province.State, }
\NormalTok{                  Country.Region, }\FunctionTok{ncol}\NormalTok{(deaths\_df))}
\FunctionTok{names}\NormalTok{(deaths\_df)[}\DecValTok{3}\NormalTok{] }\OtherTok{\textless{}{-}} \StringTok{"deaths"}
\NormalTok{recovered\_df}\OtherTok{\textless{}{-}}\FunctionTok{read.csv}\NormalTok{(}\StringTok{"data/time\_series\_covid19\_recovered\_global.csv"}\NormalTok{, }
                       \AttributeTok{header=}\ConstantTok{TRUE}\NormalTok{, }\AttributeTok{stringsAsFactors=}\ConstantTok{FALSE}\NormalTok{)}
\NormalTok{recovered\_df}\OtherTok{\textless{}{-}}\FunctionTok{select}\NormalTok{(recovered\_df, Province.State, }
\NormalTok{                     Country.Region, }\FunctionTok{ncol}\NormalTok{(recovered\_df))}
\FunctionTok{names}\NormalTok{(recovered\_df)[}\DecValTok{3}\NormalTok{] }\OtherTok{\textless{}{-}} \StringTok{"recovered"}
\CommentTok{\# Combine the datasets into one and fill NA with 0}
\NormalTok{combined\_df}\OtherTok{\textless{}{-}}\FunctionTok{full\_join}\NormalTok{(confirmed\_df, recovered\_df, }
                       \AttributeTok{by=}\FunctionTok{c}\NormalTok{(}\StringTok{"Province.State"}\NormalTok{, }\StringTok{"Country.Region"}\NormalTok{))}
\NormalTok{combined\_df}\OtherTok{\textless{}{-}}\FunctionTok{full\_join}\NormalTok{(combined\_df, deaths\_df, }
                       \AttributeTok{by=}\FunctionTok{c}\NormalTok{(}\StringTok{"Province.State"}\NormalTok{, }\StringTok{"Country.Region"}\NormalTok{))}
\NormalTok{combined\_df[}\FunctionTok{is.na}\NormalTok{(combined\_df)] }\OtherTok{\textless{}{-}} \DecValTok{0}
\CommentTok{\# Assignment is unclear if we are to consider state and region or }
\CommentTok{\# just region. Based on how data is formatted, I think it is cleaner }
\CommentTok{\# and makes more sence to use region only.  For instance, in confirmed }
\CommentTok{\# dataset, Canada is broken up by region, but in recovered dataset it }
\CommentTok{\# uses Canada as a whole.  There are numerous examples of this in}
\CommentTok{\# the data}
\NormalTok{grouped\_df}\OtherTok{\textless{}{-}}\FunctionTok{as.data.frame}\NormalTok{(}\FunctionTok{summarise\_each}\NormalTok{(}\FunctionTok{group\_by}\NormalTok{(}
  \FunctionTok{select}\NormalTok{(combined\_df, }\SpecialCharTok{{-}}\NormalTok{Province.State), Country.Region), sum))}
\end{Highlighting}
\end{Shaded}

\begin{verbatim}
## Warning: `summarise_each_()` was deprecated in dplyr 0.7.0.
## i Please use `across()` instead.
## i The deprecated feature was likely used in the dplyr package.
##   Please report the issue at <]8;;https://github.com/tidyverse/dplyr/issueshttps://github.com/tidyverse/dplyr/issues]8;;>.
## This warning is displayed once every 8 hours.
## Call `lifecycle::last_lifecycle_warnings()` to see where this warning was
## generated.
\end{verbatim}

\begin{Shaded}
\begin{Highlighting}[]
\CommentTok{\# compute risk and burden by region}
\NormalTok{grouped\_df}\SpecialCharTok{$}\NormalTok{risk}\OtherTok{\textless{}{-}}\NormalTok{grouped\_df}\SpecialCharTok{$}\NormalTok{deaths }\SpecialCharTok{/}\NormalTok{ grouped\_df}\SpecialCharTok{$}\NormalTok{recovered}
\NormalTok{grouped\_df}\SpecialCharTok{$}\NormalTok{burden}\OtherTok{\textless{}{-}}\NormalTok{grouped\_df}\SpecialCharTok{$}\NormalTok{confirmed }\SpecialCharTok{*}\NormalTok{ grouped\_df}\SpecialCharTok{$}\NormalTok{risk}
\FunctionTok{cat}\NormalTok{(}\StringTok{"Highest risk scores"}\NormalTok{)}
\end{Highlighting}
\end{Shaded}

\begin{verbatim}
## Highest risk scores
\end{verbatim}

\begin{Shaded}
\begin{Highlighting}[]
\FunctionTok{head}\NormalTok{(}\FunctionTok{arrange}\NormalTok{(grouped\_df, }\SpecialCharTok{{-}}\NormalTok{risk, }\SpecialCharTok{{-}}\NormalTok{confirmed))}
\end{Highlighting}
\end{Shaded}

\begin{verbatim}
##   Country.Region confirmed recovered  deaths risk burden
## 1             US 103802702         0 1123836  Inf    Inf
## 2          India  44690738         0  530779  Inf    Inf
## 3         France  39866718         0  166176  Inf    Inf
## 4        Germany  38249060         0  168935  Inf    Inf
## 5         Brazil  37076053         0  699276  Inf    Inf
## 6          Japan  33320438         0   72997  Inf    Inf
\end{verbatim}

\begin{Shaded}
\begin{Highlighting}[]
\FunctionTok{cat}\NormalTok{(}\StringTok{"Highest risk scores, that are not infinite"}\NormalTok{)}
\end{Highlighting}
\end{Shaded}

\begin{verbatim}
## Highest risk scores, that are not infinite
\end{verbatim}

\begin{Shaded}
\begin{Highlighting}[]
\FunctionTok{head}\NormalTok{(}\FunctionTok{arrange}\NormalTok{(grouped\_df[grouped\_df}\SpecialCharTok{$}\NormalTok{risk }\SpecialCharTok{!=} \ConstantTok{Inf}\NormalTok{,], }\SpecialCharTok{{-}}\NormalTok{risk, }\SpecialCharTok{{-}}\NormalTok{confirmed))}
\end{Highlighting}
\end{Shaded}

\begin{verbatim}
##      Country.Region confirmed recovered deaths risk burden
## NA             <NA>        NA        NA     NA   NA     NA
## NA.1           <NA>        NA        NA     NA   NA     NA
## NA.2           <NA>        NA        NA     NA   NA     NA
## NA.3           <NA>        NA        NA     NA   NA     NA
## NA.4           <NA>        NA        NA     NA   NA     NA
\end{verbatim}

\begin{Shaded}
\begin{Highlighting}[]
\FunctionTok{cat}\NormalTok{(}\StringTok{"Lowest Risk Scores"}\NormalTok{)}
\end{Highlighting}
\end{Shaded}

\begin{verbatim}
## Lowest Risk Scores
\end{verbatim}

\begin{Shaded}
\begin{Highlighting}[]
\FunctionTok{head}\NormalTok{(}\FunctionTok{arrange}\NormalTok{(grouped\_df, risk, confirmed))}
\end{Highlighting}
\end{Shaded}

\begin{verbatim}
##     Country.Region confirmed recovered deaths risk burden
## 1     Korea, North         1         0      6  Inf    Inf
## 2       MS Zaandam         9         0      2  Inf    Inf
## 3 Diamond Princess       712         0     13  Inf    Inf
## 4         Kiribati      5014         0     18  Inf    Inf
## 5            Nauru      5247         0      1  Inf    Inf
## 6            Palau      5991         0      9  Inf    Inf
\end{verbatim}

\begin{Shaded}
\begin{Highlighting}[]
\FunctionTok{cat}\NormalTok{(}\StringTok{"Lowest risk scores, that are not zero"}\NormalTok{)}
\end{Highlighting}
\end{Shaded}

\begin{verbatim}
## Lowest risk scores, that are not zero
\end{verbatim}

\begin{Shaded}
\begin{Highlighting}[]
\FunctionTok{head}\NormalTok{(}\FunctionTok{arrange}\NormalTok{(grouped\_df[grouped\_df}\SpecialCharTok{$}\NormalTok{risk }\SpecialCharTok{!=} \DecValTok{0}\NormalTok{,], risk, confirmed))}
\end{Highlighting}
\end{Shaded}

\begin{verbatim}
##       Country.Region confirmed recovered deaths risk burden
## 94      Korea, North         1         0      6  Inf    Inf
## 108       MS Zaandam         9         0      2  Inf    Inf
## 50  Diamond Princess       712         0     13  Inf    Inf
## 93          Kiribati      5014         0     18  Inf    Inf
## 127            Nauru      5247         0      1  Inf    Inf
## 138            Palau      5991         0      9  Inf    Inf
\end{verbatim}

\begin{Shaded}
\begin{Highlighting}[]
\NormalTok{global\_confirmed}\OtherTok{\textless{}{-}}\FunctionTok{sum}\NormalTok{(grouped\_df}\SpecialCharTok{$}\NormalTok{confirmed)}
\NormalTok{global\_deaths}\OtherTok{\textless{}{-}}\FunctionTok{sum}\NormalTok{(grouped\_df}\SpecialCharTok{$}\NormalTok{deaths)}
\NormalTok{global\_recovered}\OtherTok{\textless{}{-}}\FunctionTok{sum}\NormalTok{(grouped\_df}\SpecialCharTok{$}\NormalTok{recovered)}
\NormalTok{global\_risk}\OtherTok{\textless{}{-}}\NormalTok{global\_deaths }\SpecialCharTok{/}\NormalTok{ global\_recovered}
\NormalTok{global\_burden}\OtherTok{\textless{}{-}}\NormalTok{global\_confirmed }\SpecialCharTok{*}\NormalTok{ global\_risk}
\FunctionTok{cat}\NormalTok{(}\StringTok{"Global Data}\SpecialCharTok{\textbackslash{}n}\StringTok{"}\NormalTok{, }
    \StringTok{"Confirmed:"}\NormalTok{, global\_confirmed, }\StringTok{"}\SpecialCharTok{\textbackslash{}n}\StringTok{"}\NormalTok{, }
    \StringTok{"Deaths:   "}\NormalTok{, global\_deaths, }\StringTok{"}\SpecialCharTok{\textbackslash{}n}\StringTok{"}\NormalTok{,}
    \StringTok{"Recovered:"}\NormalTok{, global\_recovered, }\StringTok{"}\SpecialCharTok{\textbackslash{}n}\StringTok{"}\NormalTok{,}
    \StringTok{"Risk:     "}\NormalTok{, global\_risk, }\StringTok{"}\SpecialCharTok{\textbackslash{}n}\StringTok{"}\NormalTok{,}
    \StringTok{"Burden:   "}\NormalTok{, global\_burden, }\StringTok{"}\SpecialCharTok{\textbackslash{}n}\StringTok{"}\NormalTok{)}
\end{Highlighting}
\end{Shaded}

\begin{verbatim}
## Global Data
##  Confirmed: 676570149 
##  Deaths:    6881802 
##  Recovered: 0 
##  Risk:      Inf 
##  Burden:    Inf
\end{verbatim}

\hypertarget{objective-5}{%
\subsubsection{Objective 5}\label{objective-5}}

You are asked to make two tables with the top 5 countries that have the
most COVID-19 related confirmations and and deaths. Make sure to include
all of the counts for the country, not just the counts for one area in
the country. To do this we will need to sum all of the values for each
country, create new data frames from these values, and use the package
kable to convert those data frames into tables.

Hint: Sum each country's counts by subsetting the data frame using a
list of countries available in the data set. Use a for loop to iterate
through the data frame using the list of countries. For each country,
calculate the count sum and assign this value to a list.

\begin{Shaded}
\begin{Highlighting}[]
\CommentTok{\# Datasets respresent a cumlative sum by date, so last column represents }
\CommentTok{\# sumation for region}
\NormalTok{confirmed\_df}\OtherTok{\textless{}{-}}\FunctionTok{read.csv}\NormalTok{(}\StringTok{"data/time\_series\_covid19\_confirmed\_global.csv"}\NormalTok{, }
                       \AttributeTok{header=}\ConstantTok{TRUE}\NormalTok{, }\AttributeTok{stringsAsFactors=}\ConstantTok{FALSE}\NormalTok{)}
\NormalTok{confirmed\_df}\OtherTok{\textless{}{-}}\FunctionTok{select}\NormalTok{(confirmed\_df, Province.State, }
\NormalTok{                     Country.Region, }\FunctionTok{ncol}\NormalTok{(confirmed\_df))}
\FunctionTok{names}\NormalTok{(confirmed\_df)[}\DecValTok{3}\NormalTok{] }\OtherTok{\textless{}{-}} \StringTok{"confirmed"}
\NormalTok{deaths\_df}\OtherTok{\textless{}{-}}\FunctionTok{read.csv}\NormalTok{(}\StringTok{"data/time\_series\_covid19\_deaths\_global.csv"}\NormalTok{, }
                    \AttributeTok{header=}\ConstantTok{TRUE}\NormalTok{, }\AttributeTok{stringsAsFactors=}\ConstantTok{FALSE}\NormalTok{)}
\NormalTok{deaths\_df}\OtherTok{\textless{}{-}}\FunctionTok{select}\NormalTok{(deaths\_df, Province.State, }
\NormalTok{                  Country.Region, }\FunctionTok{ncol}\NormalTok{(deaths\_df))}
\FunctionTok{names}\NormalTok{(deaths\_df)[}\DecValTok{3}\NormalTok{] }\OtherTok{\textless{}{-}} \StringTok{"deaths"}
\NormalTok{recovered\_df}\OtherTok{\textless{}{-}}\FunctionTok{read.csv}\NormalTok{(}\StringTok{"data/time\_series\_covid19\_recovered\_global.csv"}\NormalTok{, }
                       \AttributeTok{header=}\ConstantTok{TRUE}\NormalTok{, }\AttributeTok{stringsAsFactors=}\ConstantTok{FALSE}\NormalTok{)}
\NormalTok{recovered\_df}\OtherTok{\textless{}{-}}\FunctionTok{select}\NormalTok{(recovered\_df, Province.State, }
\NormalTok{                     Country.Region, }\FunctionTok{ncol}\NormalTok{(recovered\_df))}
\FunctionTok{names}\NormalTok{(recovered\_df)[}\DecValTok{3}\NormalTok{] }\OtherTok{\textless{}{-}} \StringTok{"recovered"}
\CommentTok{\# Combine the datasets into one and fill NA with 0}
\NormalTok{combined\_df}\OtherTok{\textless{}{-}}\FunctionTok{full\_join}\NormalTok{(confirmed\_df, recovered\_df, }
                       \AttributeTok{by=}\FunctionTok{c}\NormalTok{(}\StringTok{"Province.State"}\NormalTok{, }\StringTok{"Country.Region"}\NormalTok{))}
\NormalTok{combined\_df}\OtherTok{\textless{}{-}}\FunctionTok{full\_join}\NormalTok{(combined\_df, deaths\_df, }
                       \AttributeTok{by=}\FunctionTok{c}\NormalTok{(}\StringTok{"Province.State"}\NormalTok{, }\StringTok{"Country.Region"}\NormalTok{))}
\NormalTok{combined\_df[}\FunctionTok{is.na}\NormalTok{(combined\_df)] }\OtherTok{\textless{}{-}} \DecValTok{0}
\CommentTok{\# Group and combine data by region}
\NormalTok{grouped\_df}\OtherTok{\textless{}{-}}\FunctionTok{as.data.frame}\NormalTok{(}\FunctionTok{summarise\_each}\NormalTok{(}\FunctionTok{group\_by}\NormalTok{(}
  \FunctionTok{select}\NormalTok{(combined\_df, }\SpecialCharTok{{-}}\NormalTok{Province.State), Country.Region), sum))}
\CommentTok{\# compute risk and burden by region}
\NormalTok{grouped\_df}\SpecialCharTok{$}\NormalTok{risk}\OtherTok{\textless{}{-}}\NormalTok{grouped\_df}\SpecialCharTok{$}\NormalTok{deaths }\SpecialCharTok{/}\NormalTok{ grouped\_df}\SpecialCharTok{$}\NormalTok{recovered}
\NormalTok{grouped\_df}\SpecialCharTok{$}\NormalTok{burden}\OtherTok{\textless{}{-}}\NormalTok{grouped\_df}\SpecialCharTok{$}\NormalTok{confirmed }\SpecialCharTok{*}\NormalTok{ grouped\_df}\SpecialCharTok{$}\NormalTok{risk}
\NormalTok{confirmed\_tb }\OtherTok{=} \FunctionTok{kable}\NormalTok{(}\FunctionTok{arrange}\NormalTok{(grouped\_df, }\SpecialCharTok{{-}}\NormalTok{confirmed)[}\DecValTok{1}\SpecialCharTok{:}\DecValTok{5}\NormalTok{,])}
\NormalTok{deaths\_tb }\OtherTok{=} \FunctionTok{kable}\NormalTok{(}\FunctionTok{arrange}\NormalTok{(grouped\_df, }\SpecialCharTok{{-}}\NormalTok{deaths)[}\DecValTok{1}\SpecialCharTok{:}\DecValTok{5}\NormalTok{,])}
\NormalTok{recovered\_tb }\OtherTok{=} \FunctionTok{kable}\NormalTok{(}\FunctionTok{arrange}\NormalTok{(grouped\_df, }\SpecialCharTok{{-}}\NormalTok{recovered)[}\DecValTok{1}\SpecialCharTok{:}\DecValTok{5}\NormalTok{,])}
\FunctionTok{cat}\NormalTok{(}\StringTok{"Top 5 confirmed regions"}\NormalTok{)}
\end{Highlighting}
\end{Shaded}

\begin{verbatim}
## Top 5 confirmed regions
\end{verbatim}

\begin{Shaded}
\begin{Highlighting}[]
\NormalTok{confirmed\_tb}
\end{Highlighting}
\end{Shaded}

\begin{tabular}{l|r|r|r|r|r}
\hline
Country.Region & confirmed & recovered & deaths & risk & burden\\
\hline
US & 103802702 & 0 & 1123836 & Inf & Inf\\
\hline
India & 44690738 & 0 & 530779 & Inf & Inf\\
\hline
France & 39866718 & 0 & 166176 & Inf & Inf\\
\hline
Germany & 38249060 & 0 & 168935 & Inf & Inf\\
\hline
Brazil & 37076053 & 0 & 699276 & Inf & Inf\\
\hline
\end{tabular}

\begin{Shaded}
\begin{Highlighting}[]
\FunctionTok{cat}\NormalTok{(}\StringTok{"Top 5 deaths regions"}\NormalTok{)}
\end{Highlighting}
\end{Shaded}

\begin{verbatim}
## Top 5 deaths regions
\end{verbatim}

\begin{Shaded}
\begin{Highlighting}[]
\NormalTok{deaths\_tb}
\end{Highlighting}
\end{Shaded}

\begin{tabular}{l|r|r|r|r|r}
\hline
Country.Region & confirmed & recovered & deaths & risk & burden\\
\hline
US & 103802702 & 0 & 1123836 & Inf & Inf\\
\hline
Brazil & 37076053 & 0 & 699276 & Inf & Inf\\
\hline
India & 44690738 & 0 & 530779 & Inf & Inf\\
\hline
Russia & 22075858 & 0 & 388478 & Inf & Inf\\
\hline
Mexico & 7483444 & 0 & 333188 & Inf & Inf\\
\hline
\end{tabular}

\begin{Shaded}
\begin{Highlighting}[]
\FunctionTok{cat}\NormalTok{(}\StringTok{"Top 5 recovered regions"}\NormalTok{)}
\end{Highlighting}
\end{Shaded}

\begin{verbatim}
## Top 5 recovered regions
\end{verbatim}

\begin{Shaded}
\begin{Highlighting}[]
\NormalTok{recovered\_tb}
\end{Highlighting}
\end{Shaded}

\begin{tabular}{l|r|r|r|r|r}
\hline
Country.Region & confirmed & recovered & deaths & risk & burden\\
\hline
Afghanistan & 209451 & 0 & 7896 & Inf & Inf\\
\hline
Albania & 334457 & 0 & 3598 & Inf & Inf\\
\hline
Algeria & 271496 & 0 & 6881 & Inf & Inf\\
\hline
Andorra & 47890 & 0 & 165 & Inf & Inf\\
\hline
Angola & 105288 & 0 & 1933 & Inf & Inf\\
\hline
\end{tabular}

\hypertarget{github-log}{%
\subsubsection{GitHub Log}\label{github-log}}

\texttt{\#\{bash\ gitlog\}\ \ \#git\ log\ -\/-pretty=format:"\%nSubject:\ \%s\%nAuthor:\ \%aN\%nDate:\ \%aD\%nBody:\ \%b"\ \#\{r\ ob5\}}

\texttt{\#\{r\ cars\}\ \#summary(cars)}

\begin{verbatim}
## Including Plots

You can also embed plots, for example:
\end{verbatim}

\texttt{\#\{r\ pressure,\ echo=FALSE\}\ plot(pressure)}

\begin{verbatim}
Note that the `echo = FALSE` parameter was added to the code chunk to prevent printing of the R code that generated the plot.
\end{verbatim}

\end{document}
